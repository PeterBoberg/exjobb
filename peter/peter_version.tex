\documentclass[12pt, a4paper, onecolumn]{article}
\usepackage{fontspec}
\usepackage{titlesec}
\usepackage{tocloft}
\usepackage[english]{babel}
\usepackage{blindtext}
\usepackage{subfig}
\usepackage{pgf}
\setmainfont{Georgia}
\usepackage{parskip}
\usepackage{float}

\newcommand\sectionfont{\normalfont\fontspec{Arial}\fontsize{14pt}{0}\bfseries}
\newcommand\subsectionfont{\normalfont\fontspec{Arial}\fontsize{13pt}{0}\bfseries}
\newcommand\subsubsectionfont{\normalfont\fontspec{Arial}\fontsize{12pt}{0}\bfseries}
\newcommand\tocsectionfont{\normalfont\fontspec{Arial}\fontsize{12pt}{0}\bfseries}
\newcommand\tocsubsectionfont{\normalfont\fontspec{Arial}\fontsize{11pt}{0}\bfseries}
\newcommand\tocsubsubsectionfont{\normalfont\fontspec{Arial}\fontsize{11pt}{0}}
\newcommand\toctitlefont{\normalfont\fontspec{Arial}\fontsize{16pt}{0}\bfseries}

\titleformat{\section}{\sectionfont}{\thesection}{20pt}{}
\titleformat{\subsection}{\subsectionfont}{\thesubsection}{20pt}{}
\titleformat{\subsubsection}{\subsubsectionfont}{\thesubsubsection}{20pt}{}

\renewcommand{\cftsecfont}{\tocsectionfont}
\renewcommand{\cftsubsecfont}{\tocsubsectionfont}
\renewcommand{\cftsubsubsecfont}{\tocsubsubsectionfont}
\renewcommand{\cftsecpagefont}{\tocsectionfont}
\renewcommand{\cftsubsecpagefont}{\tocsubsectionfont}
\renewcommand{\cftsubsubsecpagefont}{\tocsubsubsectionfont}
\renewcommand{\cfttoctitlefont}{\toctitlefont}

\newcommand{\parag}[1]{
	\textbf{#1} \hspace{0pt} \\
}

\addto\captionsenglish{
	\renewcommand{\contentsname}{Table of Contents}
}

\begin{document}

	\section{Fall Detection Application: Evaluation}
		Evaluation of the application was done by conducting a series of test in a real-world setting. The test aims to evaluate two things, the accurateness of the fall detection system, and the impact on battery this system implies. All tests scenarios were identical on the iOS and Android platform. 
		
		In the case of accurateness, the measurements are expressed in terms of \textit{false positives, true positives, false negatives and true negatives}.Depending on the test, we continuously strived for either true negatives or true positives, since these terms states that the outcome of the test is also what were expected  beforehand (\textit{either true or false}). The other two terms implies the opposite, that the outcome was reversed to what was expected (\textit{either true or false}).
		
	
		
	\subsection{Evaluating accuracy} 
		When evaluating the accurateness of the fall detection algorithm we constructed a total of seven tests cases that would mimic real life situations that might trigger an alarm. Each test case was repeated 10 times on an iOS and Android device respectively. The tests were performed outdoors under equal circumstances. Test cases were as follows:
			\begin{itemize}
				\item Walk.
				\item Run.
				\item Jump.
				\item Falling while walking.
				\item Falling from 1 m.
				\item Falling from 1,5 m.
				\item Falling from 3 m.
			\end{itemize}
		
		
	The results from each of these test cases will now be presented in chronological order. 
	
		\subsubsection{Walk} 
		The walking test was performed by simply walking with the device in the trouser pocket. We walked for 15 meters and stopped for 10 seconds, then repeated the process. The 10 second stop is definitely long enough for a possible motion-event to fall through the state machine and consequently consult the classification engine. Should that happen, we still expect the classification engine to not classify the event as a fall. Walking should not produce an alarm and we expect the fall detection system to discard all motion-events and thus get \textit{true negatives}. The result can be seen in table \ref{table:walk-test}. 
		
			\begin{table}[H]
				\centering
				\begin{tabular}{|l|c|c|c|}
					\hline
					& True negative & False positive & Success rate \\ \hline
					iOS     & 10            & 0              & 100\%        \\ \hline
					Android & 10            & 0              & 100\%        \\ \hline
				\end{tabular}
					\caption{The result of the walking test}
					\label{table:walk-test}
			\end{table}
		
		The result of the walking test shows a 100\% success rate on both iOS and Android. Possible fall event candidates are successfully discarded and walking does not produce an alarm.
		
		
		\subsubsection{Run}
		The run test was performed similar to the walking test in 1.1.1 but instead of walking, we would run for 15 meters and then stop for 10 seconds. Like the walking test, we expect the system to correctly discard all motion-events as non-fall. Running, like walking, should not produce any alarms and thus we expect to get \textit{true negatives}. The result can be see in table \ref{table:run-test}
		
		
		\begin{table}[H]
			\centering
			\begin{tabular}{|l|c|c|c|}
				\hline
				& True negative & False positive & Success rate \\ \hline
				iOS     & 10            & 0              & 100\%        \\ \hline
				Android & 10            & 0              & 100\%        \\ \hline
			\end{tabular}
			\caption{The result of the running test}
			\label{table:run-test}
		\end{table}
		
		The result of the running test shows that equal to the walking test, the system successfully discards all motion-events as non-fall and does not produce an alarm.
		
		
		\subsubsection{Jump}
		The jump test was done by keeping the device in the trouser pocket and jump from a stare case elevated by approximately 40 cm and land on the feet on solid asphalt. This test, like the walk and run test, should preferably not produce any alarms. However, since the acceleration curve for a jump is very similar to that of a fall, we anticipated this test to be quite a challenge for our system since every jump would certainly slip through the state machine and get classified. The only barrier left is thus the classification engine who hopefully would classify each tests as a non-fall. We expected to get \textit{true negatives}. The results can be seen in table \ref{table:jump-test}.
		
			\begin{table}[H]
			\centering
			\begin{tabular}{|l|c|c|c|}
				\hline
				& True negative & False positive & Success rate \\ \hline
				iOS     & 9            & 1             & 90\%        \\ \hline
				Android & 9            & 1              & 90\%        \\ \hline
			\end{tabular}
			\caption{The result of the jumping test}
			\label{table:jump-test}
		\end{table}
	
		The result of the jumping test shows that the system successfully discards 90\% of the jumps as non-falls on both devices. Only a single test per device were misclassified and resulted in an alarm.
		
		\subsubsection{Falling while walking}
		\label{section:falling-while-walking}
		We tried to simulate falling while walking by holding the device tightly against the hip and then walk for 15m and drop the device to the ground while still walking. The device would thus produce a curved fall to the ground in contrast to a straight line the would be the result if a person would fall from for eg. a ladder. In this test we expected the application to produce alarms, since falling while walking is indeed still a fall. Thus, we want the number of \textit{true-positives} be high and the number of \textit{false-negatives} to be low. The results can be seen in table \ref{table:fall-while-walk}
		
			\begin{table}[H]
				\centering
				\begin{tabular}{|l|c|c|c|}
					\hline
					& True positive & False negative & Success rate \\ \hline
					iOS     & 9            & 1             & 90\%        \\ \hline
					Android & 5            & 5              & 50\%        \\ \hline
				\end{tabular}
				\caption{The result of the falling while walking test}
				\label{table:fall-while-walk}
			\end{table}

		As can be seen, the iOS application successfully recognizes the event as a fall 90\% of the times where as the Android application had some difficulties to do so in 50\% of the times.
		
		
		\subsubsection{Falling  from 1m ( Simulate falling while standing on the ground)}
		\label{section:falling-from-1m}
		The  \textit{falling from 1m} test was performed by simply dropping the device to the ground from 1m altitude. The aim of this test was to replicate if a person would fall while standing still on the ground and having the device in the trouser pockets. Of course we were expecting the application to produce alarm in this case to the greatest extent. Thus, similar to the \textit{Falling while walking} test described in section \ref{section:falling-while-walking}, we anticipated the number if \textit{true-positives} to be high and the number of \textit{false-negatives} to be low. The result can be seen in table \ref{table:fall-from-1m}
		
			\begin{table}[H]
				\centering
				\begin{tabular}{|l|c|c|c|}
					\hline
					& True positive & False negative & Success rate \\ \hline
					iOS     & 8            & 2             & 80\%        \\ \hline
					Android & 7            & 3              & 70\%        \\ \hline
				\end{tabular}
				\caption{The result of the fall from 1m test}
				\label{table:fall-from-1m}
			\end{table}
	
		The test result shows that the iOS application successfully detects falls 80\% of the time where as the Android application had a success rate of 70\%. Once again, the iOS application has a slightly higher success rate,, but both of the applications detects falls in a vast majority of the tests. The reason for not getting a 100\% \textit{true-positives} might be due to that falling while standing on the ground does not produce a significant impact leading to the state machine firing.
		
		\subsubsection{Falling from 1.5 meters (Simulate falling from eg. a chair)}
		\label{section:falling-from-1.5m}
		The \textit{falling from 1.5m} test, was performed in a similar fashion to \textit{Falling from 1m} in section \ref{section:falling-from-1m}, the only difference was that the device was elevated by another 0.5 meters. The aim of this test was to replicate a person falling from eg. a chair or some other common slightly elevated house hold apparel. In this test we expect the application to successfully detect falls on all occasions, thus, we anticipate the number of \textit{true-positives} to be high and the number of \textit{false-negatives to be low}. The result can be seen in table \ref{table:fall-from-1.5m}
		
				\begin{table}[H]
			\centering
			\begin{tabular}{|l|c|c|c|}
				\hline
				& True positive & False negative & Success rate \\ \hline
				iOS     & 9            & 1             & 90\%        \\ \hline
				Android & 7            & 3              & 70\%        \\ \hline
			\end{tabular}
			\caption{The result of the fall from 1.5m test}
			\label{table:fall-from-1.5m}
		\end{table}
	
		The result shown that the iOS application successfully detects fall in 90\% of the time where as the Android application has a success rate of 70\%
		
		
		\subsubsection{Falling from 3m (Simulate falling from a ladder)}
		The \textit{falling from 3m} test was conducted by letting the device fall freely from an altitude of 3m. The aim of this test was to simulate a fall from a higher elevation, namely a ladder or something similar. A fall from an altitude of 3m should definitely trigger an alarm and thus, like the tests in \ref{section:falling-from-1m} and \ref{section:falling-from-1.5m} we wanted to get as many \textit{true-positives} as possible, thus minimizing the number of \textit{false-negatives}. The results can be seen in table \ref{table:fall-from-3m}
		
		\begin{table}[H]
			\centering
			\begin{tabular}{|l|c|c|c|}
				\hline
				& True positive & False negative & Success rate \\ \hline
				iOS     & 9            & 1             & 90\%        \\ \hline
				Android & 10           & 0             & 100\%        \\ \hline
			\end{tabular}
			\caption{The result of the fall from 3m test}
			\label{table:fall-from-3m}
		\end{table}
	
			The result shown that the iOS application successfully detects fall in 90\% of the time where as the Android application has a success rate of 100\%
			
			
			
		\subsection{Summarizing accuracy}
		The overall results show that there exists some differences between the application running on an iOS device and it´s Android counterpart. For each device, we can calculate the overall success rate by the equation in \ref{equation:overall-success}
		
		\begin{equation}
			\label{equation:overall-success}
			\sum_{k=1}^{n} S_{k} / n
		\end{equation}
		
		where:
		\begin{itemize}
			\item $n$ it the number ot test cases (7 in total)
			\item  $S$ is the success rate for that test case
		\end{itemize}
			
		
		This gives us overall success rate: 
		
		iOS: $(100 + 100 + 90 + 90 + 80 + 90 + 90) / 7 = 0.914 = 91\%$
		
		
		Android: $(100 +100 +90 +50 +70  + 70 + 100) / 7 = 0.828 = 83\%$
		
		
		One reason for the discrepancies between the application running on the iOS device and then one running on Android might be that the hardware accelerometer in the iOS device is of better quality and has better accuracy and precision. 
		
\end{document}