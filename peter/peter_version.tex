\documentclass[12pt, a4paper, onecolumn]{article}
\usepackage{fontspec}
\usepackage{titlesec}
\usepackage[english]{babel}
\usepackage{blindtext}
\usepackage{subfig}
\usepackage{pgf}

\setmainfont{Georgia}
\titleformat{\section}
{\normalfont\fontspec{Arial}\fontsize{20pt}{0}\bfseries}
{\thesection}{20pt}{}

\begin{document}

\section{Theoretical Background}
	\subsection{Related works} Two related works have been researched prior as means of background for this thesis. The first one is a paper produced by Yidirim et al. \cite{int_journ} and the other one is a report produced by Abbatea et al. \cite{piza_uni}. The both aim to produce an application embedded on an Android device with capabilities to detect fall accidents. They both rely on accelerometer data from the device as a means of detecting a fall but the differ significantly in how they use that data. 
	
	\subsubsection{Yildirim et al.}
	Yildrim et al. \cite{int_journ} uses a simple threshold-based algorithm to detect falls. When an acceleration-peak exceeding a threshold value, followed by a drop in acceleration below another threshold value has occurred within a limited time window, their application flags the event a possible fall and presents a user interface. This user interface lets the user discard the detected fall if it was a false alarm. If the user does not discard the detected fall, the application will send an SMS to predefined contacts, informing them of the event. 
	
	They tested the accuracy of their algorithm, with regards to distinguish an actual fall from other activities of daily living in terms of \textit{False positives} and \textit{false negatives}. In their tests, they used a total of five people, testing different scenarios five times each, the scenarios were:
	
		\begin{itemize}
			\item Falling
			\item Walking
			\item Sitting
			\item Jumping
			\item Lying
			\item Climbing stairs
		\end{itemize}
	
	The result shows that the number of \textit{false positives} were 9/25 on the falls scenario, and the number of \textit{false positives} were 7/25 on the jumping scenario. This means that their algorithm missed 9 out of 25 actual falls, and falsely flagged 7 out of 25 jumps as falls. In all other scenarios the number of \textit{false negatives/false positives} was zero. They conclude by saying that their algorithm had a hard time distinguishing between falling and jumping, but performed good in the other scenarios. 
	
	
	\subsubsection{Abbatea et al.}
		Abbatea et al. uses a significantly more sophisticated approach to detect and distinguish fall from activities of daily living. Their work involves a finite state machine as a first stop to detect....[To be continued].
		
		
		\subsection{Architecture outline}
			The overall architecture of the application consists of several steps. All readings from the accelerometer is continuously pushed to a ring buffer. This ring buffer holds the last 8 seconds of accelerometer data. On top of this, a finite state machine is used as a first barrier to filter out the most common movements of the device. If a fall-like event were to manipulate the state machine into its final state, the ring buffer is transferred over to a \texttt{Feature Extractor}. In this step, the buffer is analyzed and a total of 8 "features" are extracted (as similar to the Abbatea et el approach). These features represent important aspects of the movement of the device for the last 8 seconds. 
			
			The 8 features are then put inside a data-object and fed into a \texttt{Classification Engine} which consists of a neural network. This classification engine have the responsibility to classify the event based on the given features. The output classes of the neural network are; \textit{Fall}, \textit{Jump} and \textit{Run/Walk}. The neural network calculates the independent probability of the event's belonging to each of the three classes. If the probability of the event belonging to the \textit{Fall} class exceeds the other two, the event is considered a fall. 
			
			In the case of an event being classified as a fall, the application will present a screen along with a sonic alarm signal. The user will then have 45 seconds to discard the event or a message will be sent to a set of predefined contacts, notifying them about what´s happened.
			
			\paragraph{Insert image here}
	
	
		\subsection{Using the accelerometer}
			Using the accelerometer on the Android and iOS system is similar. Updates from the accelerometer are pushed to listening code that acts upon those updates. Updates come in the form of an data-object containing information about the device's acceleration on all three axes (\textit{X}, \textit{Y} and \textit{Z}) at that moment in time. The accelerometer's update frequency is configurable on both Android and iOS devices and we chose an update frequency of \textit{50 Hz} since it provides a reasonable balance between accuracy in measurements and battery consumption. On each update, the resultant vector is calculated according to the formula in \ref{vectorlength} and the result is put in to the ring buffer. Given this configuration, the ring buffer have a fixed length of \(50 Hz * 8 seconds = 400 samples\). 
			
			On each new update, the resultant vector is also given to the finite state machine who will act according to the description in the next section.
	

		\subsection{Implementing a neural network}
			The implementation of the neural network consists of three steps. The first step involves designing the network in terms on layers and neurons, this will from now on be referenced as \textit{the model}. The second step is related to data collection, as well as fitting and evaluating the model to make as accurate predictions as possible. The final step using the model in the application and evaluate it accurateness in real world scenarios. 

		\subsubsection{Designing a model}
			The neural network is built using \textit{Tensor Flow}, \textit{Keras} and the \textit{Python} programing language. 

		\subsubsection{Training the model}

		\subsubsection{Using the model}

\bibliography{bib_peter}
\bibliographystyle{ieeetr}

\end{document}