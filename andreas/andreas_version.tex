\documentclass[12pt, a4paper, onecolumn]{article}
\usepackage{fontspec}
\usepackage{titlesec}
\usepackage{tocloft}
\usepackage[english]{babel}
\usepackage{blindtext}
\usepackage{subfig}
\usepackage{pgf}
\setmainfont{Georgia}
\usepackage{parskip}
\usepackage{float}

\newcommand\sectionfont{\normalfont\fontspec{Arial}\fontsize{14pt}{0}\bfseries}
\newcommand\subsectionfont{\normalfont\fontspec{Arial}\fontsize{13pt}{0}\bfseries}
\newcommand\subsubsectionfont{\normalfont\fontspec{Arial}\fontsize{12pt}{0}\bfseries}
\newcommand\tocsectionfont{\normalfont\fontspec{Arial}\fontsize{12pt}{0}\bfseries}
\newcommand\tocsubsectionfont{\normalfont\fontspec{Arial}\fontsize{11pt}{0}\bfseries}
\newcommand\tocsubsubsectionfont{\normalfont\fontspec{Arial}\fontsize{11pt}{0}}
\newcommand\toctitlefont{\normalfont\fontspec{Arial}\fontsize{16pt}{0}\bfseries}

\titleformat{\section}{\sectionfont}{\thesection}{20pt}{}
\titleformat{\subsection}{\subsectionfont}{\thesubsection}{20pt}{}
\titleformat{\subsubsection}{\subsubsectionfont}{\thesubsubsection}{20pt}{}

\renewcommand{\cftsecfont}{\tocsectionfont}
\renewcommand{\cftsubsecfont}{\tocsubsectionfont}
\renewcommand{\cftsubsubsecfont}{\tocsubsubsectionfont}
\renewcommand{\cftsecpagefont}{\tocsectionfont}
\renewcommand{\cftsubsecpagefont}{\tocsubsectionfont}
\renewcommand{\cftsubsubsecpagefont}{\tocsubsubsectionfont}
\renewcommand{\cfttoctitlefont}{\toctitlefont}

\newcommand{\parag}[1]{
	\textbf{#1} \hspace{0pt} \\
}

\addto\captionsenglish{
	\renewcommand{\contentsname}{Table of Contents}
}

\begin{document}
	
	\title{Fall detection using smart phone application}
	\maketitle
	
	\tableofcontents
	
	\newpage
	
	\section{Fall Detection Application: Evaluation}
	\newpage
	
	\section{Discussion}
	
	In this chapter, the methods used in the study and their implications are analysed, the research questions are answered followed by a discussion of sustainability and ethical considerations.
	
	\subsection{Methods \& Implications}
	
	This section analyses the methods used and the validity and reliability of this study.
	
	\subsubsection{Literature study}
	
	Since very similar studies had already been conducted before we started this research project, it was necessary to perform a literature study before starting the development process. Without this literature study we would not have known how previous researchers had reasoned and it would have been likely that we had repeated old mistakes and ended up with a lesser application.
	
	\subsubsection{Case study: Development}
	
	The case study consisting of developing an application provided
	
	\subsubsection{Evaluation}
	
	\subsubsection{Validity \& Reliability}
	
	This section discusses the validity and reliability of the study.
	
	\parag{Validity}
	The term \textit{validity} is referring to whether or not the research methods are measuring what the study is trying to answer. With respect to the first research question, RQ1, regarding how a fall detection application can be created, it should be clear that the case study is addressing this question since it is exactly what is done in the case study. Regarding the second question, RQ2, it is not as obvious. To answer how this kind of application affects battery life, you could argue that it would be better to measure the battery consumption of many different fall detection applications and draw conclusions based on that. This would however give pure statistical knowledge saying that application A consumes X amount of battery per hour and application B consumes Y. Without insight into how these applications function internally this kind of information says nothing -- it would not be possible to say why application A consumes more power than B, or the other way around. To perform such a comparison and to say why one application consumes more power than the other, it would be necessary to have the source code to the different applications, something that we did not have. Therefore this study can only say that if an application is developed the way that we did, you can expect that the battery consumption will be similar to the consumption in our measurements.
	
	\parag{Reliability}
	\textit{Reliability} refers to how replicable or repeatable the research project is. Another research project that asks the same questions and and uses the same methods should ideally come to the same conclusions. Our first method was to perform a literature study with the intention to find a good algorithm that were to be used as a starting point for the case study. It should be noted that another researcher, with different background and experiences may have found another algorithm as a better starting point and ended up with an application that implements fall detection entirely different, and thus could have ended up with different conclusions. However, if another researcher would use the same starting point as we did, and make the same decisions along the way, s/he should end up with the same result, and therefore the work is repeatable. 
	
	
	\subsection{Revisiting research questions}
	
	\subsection{Sustainability \& Ethics}
	
	\subsubsection{Sustainability}
	
	\subsubsection{Ethics}
	
	\section{Conclusions}
	\newpage
	
	\bibliography{bib_common}
	\bibliographystyle{ieeetr}
	
\end{document}