\documentclass[11pt, a4paper, onecolumn]{article}
\usepackage{fontspec}
\usepackage{titlesec}
\usepackage[english]{babel}
\usepackage{blindtext}
\setmainfont{Georgia}
\titleformat{\section}
{\normalfont\fontspec{Arial}\fontsize{20pt}{0}\bfseries}
{\thesection}{20pt}{}

\begin{document}
	
	\tableofcontents
	\newpage
	
	\section{Introduction}
	
	\subsection{Background}
	
	\subsection{The Problem}
	
	\subsection{Purpose}
	
	The purpose of the project is to examine the possibility of creating an application that can help minimize the negative effects of accidents related to falling. 
	
	\subsection{Goal}
	
	The goal of the project is to create an application that, in an effective way, can detect falling accidents and inform concerned contacts. This application should target employees in fields such as operations, construction, etc., and should run in the users cell phone. The user of the application should be able to register contact information to relatives, colleagues, etc. After that the user can activate the protection in the application by selecting the appropriate option, for example by pressing a button in the application. The user is supposed to do this before starting a critical operation such as performing work on an elevated height or similar. When the protection in the application is activated the application makes use of the device's embedded accelerometer to register changes in velocity. If the user would carry the device running the application with the protection actived, while the user would suffer a falling accident, the application would detect this and enter a warning state. In the warning state the application will notify the user that a fall has been detected and that the application soon will send an alarm to registered contacts. If the time limit for the warning state exceeds without any action from the user, an alarm will be sent to the registered contacts using for example SMS.   
	
	\subsubsection{Sustainability \& Community benefits}
	
	Since the application will not replace any existing technology it is difficult to discuss in which way the application will affect the environment. Our project will be focusing on software alone and will only use hardware that already exists, and thus it will not affect the environment in greater extent. You could however argue that since the application will run on a cell phone, that will draw more current if it runs an additional application in the background, our project will affect the environment but it is hard to say how great such an effect will be.
	
	From a social perspective we hope that the application will contribute to the society in a positive way, since the intention is that the application will help to minimize the negative consequences of accidents related to falling. Hopefully, the finished application will contribute to a better working environment for the users of the application, since an accident may be discovered earlier.  
	
	\subsubsection{Ethics}
	
	One important question when it comes to ethics is how personal data in the application will be handled. The users of the application will need to enter things like password, email address, etc., which should be regarded as sensitive data that must not be viewed by a third party. Besides this, we need to make shure that the application does not have any obvious security vulnerability that makes it possible for a malicious person or organisation to acquire the sensitive data in the application.
	
	Another important question is what guarantees the application provides to its users. Since the application is supposed to help users in case of an accident it will be sensitive if the application turns out to function worse than the users expected. The false negative cases, when the user is involved in an accident but the application fails to register this, must be kept to a minimum, but it must also be clear in the description of the application that such cases may occur.
	
	\subsection{Methodologies / Methods}
	
	
	
	\subsection{Limitations}
	
	\subsection{Disposition}
	
	\newpage
	
	\section{Theoretical Background}
	\newpage
	
	\section{Engineering-related and scientific content}
	\newpage
	
	\section{The work}
	\newpage
	
	\section{Result}
	\newpage
	
	\section{Conclusions}
	\newpage
	
	
\end{document}