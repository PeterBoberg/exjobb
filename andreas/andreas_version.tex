\documentclass[12pt, a4paper, onecolumn]{article}
\usepackage{fontspec}
\usepackage{titlesec}
\usepackage{tocloft}
\usepackage[english]{babel}
\usepackage{blindtext}
\usepackage{subfig}
\usepackage{pgf}
\setmainfont{Georgia}
\usepackage{parskip}
\usepackage{float}
\usepackage{multicol}


\newcommand\sectionfont{\normalfont\fontspec{Arial}\fontsize{14pt}{0}\bfseries}
\newcommand\subsectionfont{\normalfont\fontspec{Arial}\fontsize{13pt}{0}\bfseries}
\newcommand\subsubsectionfont{\normalfont\fontspec{Arial}\fontsize{12pt}{0}\bfseries}
\newcommand\tocsectionfont{\normalfont\fontspec{Arial}\fontsize{12pt}{0}\bfseries}
\newcommand\tocsubsectionfont{\normalfont\fontspec{Arial}\fontsize{11pt}{0}\bfseries}
\newcommand\tocsubsubsectionfont{\normalfont\fontspec{Arial}\fontsize{11pt}{0}}
\newcommand\toctitlefont{\normalfont\fontspec{Arial}\fontsize{16pt}{0}\bfseries}

\titleformat{\section}{\sectionfont}{\thesection}{20pt}{}
\titleformat{\subsection}{\subsectionfont}{\thesubsection}{20pt}{}
\titleformat{\subsubsection}{\subsubsectionfont}{\thesubsubsection}{20pt}{}

\renewcommand{\cftsecfont}{\tocsectionfont}
\renewcommand{\cftsubsecfont}{\tocsubsectionfont}
\renewcommand{\cftsubsubsecfont}{\tocsubsubsectionfont}
\renewcommand{\cftsecpagefont}{\tocsectionfont}
\renewcommand{\cftsubsecpagefont}{\tocsubsectionfont}
\renewcommand{\cftsubsubsecpagefont}{\tocsubsubsectionfont}
\renewcommand{\cfttoctitlefont}{\toctitlefont}

\newcommand{\parag}[1]{
	\textbf{#1} \hspace{0pt} \\
}

\addto\captionsenglish{
	\renewcommand{\contentsname}{Table of Contents}
}

\begin{document}
	
	\title{Fall detection using smart phone application}
	\maketitle
	
	\tableofcontents
	\newpage
	
	\section{Discussion \& Conclusions}
	
	In this chapter, the methods used in the study and their implications are analysed, the research questions are answered followed by a discussion of sustainability and ethical considerations. Lastly, the research project is summarized and suggestions for future research is presented.
	
	\subsection{Methods \& Implications}
	
	This section analyses the methods used and the validity and reliability of this study.
	
	\subsubsection{Literature study}
	
	Since very similar studies had already been conducted before we started this research project, it was necessary to perform a literature study before starting the development process. Without this literature study we would not have known how previous researchers had reasoned and it would have been likely that we had repeated old mistakes and ended up with a lesser application.
	
	\subsubsection{Case study: Development}
	
	The case study consisting of developing an application gave us the opportunity to test different approaches and ideas to see if the algorithm could be improved. Developing an application was necessary since we did not have access to the source code of any existing application.
	
	\subsubsection{Evaluation}
	
	\subsubsection{Validity \& Reliability}
	
	This section discusses the validity and reliability of the study.
	
	\parag{Validity}
	The term \textit{validity} is referring to whether or not the research methods are measuring what the study is trying to answer. With respect to the first research question, RQ1, regarding how a fall detection application can be created, it should be clear that the case study is addressing this question since it is exactly what is done in the case study. Regarding the second question, RQ2, it is not as obvious. To answer how this kind of application affects battery life, you could argue that it would be better to measure the battery consumption of many different fall detection applications and draw conclusions based on that. This would however give pure statistical knowledge saying that application A consumes X amount of battery per hour and application B consumes Y. Without insight into how these applications function internally this kind of information says nothing -- it would not be possible to say why application A consumes more power than B, or the other way around. To perform such a comparison, and to say why one application consumes more power than the other, it would be necessary to have the source code to the different applications. This was something that we did not have. Therefore this study can only say that if an application is developed the way that we did, you can expect that the battery consumption will be similar to the consumption in our measurements.
	
	\parag{Reliability}
	\textit{Reliability} refers to how replicable or repeatable the research project is. Another research project that asks the same questions and and uses the same methods should ideally come to the same conclusions. Our first method was to perform a literature study with the intention to find a good algorithm that were to be used as a starting point for the case study. It should be noted that another researcher, with different background and experiences may have found another algorithm as a better starting point and ended up with an application that implements fall detection entirely different, and thus could have ended up with different conclusions. However, if another researcher would use the same starting point as we did, and make the same decisions along the way, s/he should end up with the same result, and therefore the work is repeatable.
	
	\subsection{Revisiting research questions}
	
	This section answers the research questions.
	
	\textbf{RQ1} \textit{How can a mobile application, that detects falling accidents using modern smartphones, be created?}
	
	Based on the case study, we saw that a mobile application can be created so that it successfully detects falling accidents using the accelerometer in a modern smartphone. The case study showed that it was beneficial to use only the vector length of the accelerometer data instead of using the x, y and z components. Using a finite state machine was useful for filtering out possible falls, and a pre-trained neural network was suitable to classify these possible falls.
	
	\textbf{RQ2} \textit{How will such an application affect the battery life of the mobile phone?}
	
	Based on the evaluation of the application developed in the case study, we can see that a fall detection application does have a minor impact on the battery life of the smartphone running the application. The evaluation shows that this impact is small enough to make this kind of application useful.
	
	\subsection{Sustainability \& Ethics}
	
	\subsubsection{Sustainability}
	
	\subsubsection{Ethics}
	
	The IEEE Computer Society defines eight main principles in their \textit{Software Engineering Code of Ethics}. Each principle is divided into several more specific statements.
	This section will briefly go through some of the principles and how it applies to this project.
	
	\parag{Public}
	The first principle, called public, states that software engineers should act in accordance with the public interest. This principle states that the interest of the software engineer and the employer should be moderated with public good. Our belief is that this application will be good for the public, so it does not violate this statement. Further our belief is that the software is safe to use and does not harm the environment. Thus, this project does not violate this principle.
	
	\parag{Client and Employer}
	The second principle states that one should act to suite the interests of the client and employer and consistent with the public interest. More specific software engineers should work in the area of their competence and be honest about limitations in their education or previous experience. Hopefully this report is transparent about the authors previous experience so that this project conforms to this principle.
	
	\parag{Product}
	The principle about the product states that their products should meet the highest professional standards possible. In our case, since the developed product is a prototype in a field where a limited set of applications exist, it is hard to say what the standard should be. However, it should be clear in the report what kind of result you can expect from this kind of application.
	
	\parag{Judgement}
	The judgement principle states that software engineers should maintain independence in their professional judgement, and not engage in bribery, double billing, etc. Since we have not engaged in such an activity the project does not violate this principle.
	
	\parag{Profession}
	The profession principle states that software engineers should advance the reputation of the profession by promote public knowledge of software engineering, and obey laws regarding their work. Since this thesis is open for anyone to read this helps promote public knowledge, and as far as we know we have obeyed all laws that are applicable.
	
	\parag{Self}
	The principle called self states that software engineers should participate in lifelong learning and strive to improve their knowledge. In this project, we have certainly learnt a lot so it conforms to this principle.
	
	\subsection{Summary}
	
	The purpose of this thesis was to evaluate the possibility of creating a mobile application that utilized sensors in the smartphone to detect falling accidents. Similar studies had already been conducted, and the result of some of these studies were used as a starting point. The development started with using a finite state machine to detect basic falls. The application was later refined by introducing machine learning beyond just a state machine. The state machine was used to gather data of potential falls, the collected data was analysed and specific features in the fall data were defined. These features were used in the creation of a pre-trained neural network model and this model proved to be functioning better than just the state machine. The research project ends in the conclusion that a fall detection application can successfully be created using this approach.
	
	An important aspect in the beginning of the project was the battery life of the application. The evaluation of the application showed that when using this approach, the battery consumption was not a problem.
	
	\subsection{Future research}
	
	This theses presents one possible way to create a fall detection application for smartphones, and the evaluation of the developed application is quite primitive. It would be possible to perform research with a wider evaluation. For example, one could perform experiments where a crash test dummy were equipped with a smartphone running the application. This crash test dummy could then be dropped from different heights and in different fall like events to see how well the application performs in a real life scenario.
	
	Another possibility would be to examine another way of creating a fall detection application, perhaps using another algorithm. It would then be possible to evaluate how well that application performs compared to the approach presented in this thesis.
	
	\bibliography{bib_common}
	\bibliographystyle{ieeetr}
	
\end{document}